\chapter{Evaluation}
Completing the objectives in the introduction section is the first step of this project. In order to evaluate whether we have accomplished the objectives, first of all, to show the generality and expressibility of our languages, we will implement some algorithms like mergesort, Cooley-Tukey FFT or N-Body simulations in the high-level language and compile it to our language. Also, we will run code generation to the target low-level parallel code in C and measure the performance of generated code against sequential implementations as well as the parallel code generated by the original method used in the high-level framework. We would like to observe there will be no significant loss of efficiency or even better performance by adding one extra layer in the process of parallel code generations.

Also, not only will we benchmark the performance of execution of generated code, but also we will benchmark the performance of our tool-chain; \eg measure the compile time against different source code size.

Finally, we will focus on measuring the quality of generated code regarding the size of the generated code and its readability. We hope we will not witness an exponential growth of code size against input data.