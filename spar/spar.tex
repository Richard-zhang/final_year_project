\chapter{SPar: A session typed free monad EDSL for concurrency}
In order to generate parallel code from ParAlg, we first introduce the syntax of our intermediate language, the session-typed free monad EDSL for concurrency hosted in Haskell (SPar). SPar are compromised of two components: Core and Proc. Core is the language expressing sequential computation while Proc is a monadic language with message-passing primitives, communicating Core expression between different roles. We use a group of Proc interacting with each other to represent parallel computations. In addition, session typing the group of Proc ensure the computation is deadlock-free. 
\section{Computation: The Core EDSL}
Core is the elemental computation. The syntax of Core is mostly inspired by Alg \cite{AlgebraicMultipartyProtocol} and FunC, a demo DSL defined in the work done by \cite{svenningssonCombiningDeepShallow2015}. For this project, we choose to implement Core syntax as small as possible without sacrificing expressibility.
\subsection{Syntax}
\subsection{Representation of recursive data structures}
\subsection{Semantics}
\section{Communication: The Proc EDSL}
\subsection{Syntax}
\subsection{Representation in Haskell}
\subsection{Semantics}
\subsection{Session typing}
\section{Parallel computation: A group of Proc}
\subsection{Duality check}
\section{Conclusions}
In this section, we have introduced our intermediate language. It is friendly to use thanks to the monadic interface. In addition, communication and computation are independent in SPar. We can make Proc to parameterized by the type represent sequential computation so users can simply use their construction for sequential computation if they found Core is limited. More importantly, our strategy for parallelism is clear now. In a nutshell, we achieve parallelism by message-passing concurrency: spawning a group of threads on a multi-core CPU where each thread executes its corresponding Proc program.

Before jumping into the code generation, we will use next chapter to give an overview of some implementation challenges related to SPar first.