\chapter{Type-safe code generation from SPar}
SPar has two components: Core representing the unit of computation and Proc as a skeleton of the computation, describing the communication patterns. Naturally, the process of code generation from SPar should be divided into two parts correspondingly. We choose to make two parts independent of each other so that it's possible to swap the code generation strategy of one component without modifying another one.

The procedure of code generation is standard: transformation. We start our programs in a high level DSL and run a series of transformation to low-level DSL. SPar expressions are converted to a low-level EDSL which is then transformed to an abstract syntax tree (AST) of C (TODO cite the package). The generated code is obtained by pretty printing the AST.
\section{Instr: A low-level EDSL for channel communication}
In Proc, we have high-level actions like select, broadcast and branch abstracting implementation details, i.e variable declaration, variable assignment, channel initialization, channel communication and channel deletion. Hence, we need to define a EDSL containing instructions related to these low-level operations. We name it Instr. A Spar programs will be translated to a sequence of Instr.
\subsection{Syntax and semantic}
\begin{listing}
    \inputminted{Haskell}{codegen/instr.hs} 
    \caption{The syntax of Instr in Haskell with accompanying low-level data types}
    \label{codegen:code:instr}
\end{listing}
The definition of Instr is seen in \coref{codegen:code:instr}. \hask{Channel} is our abstract representation of Channel in Instr. It is indexed by a type a from the reified type \hask{ReprType a}. More details of this reified type will be introduced in \secref{codegen:sec:repr}. This type parameter preserves type in channel initialization hence make sure the value to be sent or received in this channel has the same type as this channel. This is necessary because for some target languages, the channel are typed. Similarly, type parameters in \hask{Exp} have the same functionality. \hask{Exp} is just a wrapper of the expression in Core. In later stages, we will take care of code generation of \hask{Exp} along with \hask{Instr}. \hask{Instr} defines the set of statements that will be generated and \hask{Exp} represents the sequential computation, which is a value that will be generated.

The semantic of 

\subsection{Representation types} \label{codegen:sec:repr}
SPar programs cannot be fully parametric since the target languages of code generation from SPar are usually less expressive, i.e, they do not treat function type \hask{a->b} as a value, and are less efficient when processing with some specific form of data, i.e, languages targeting GPUs are usually more productive in dealing with array of floating point number while slow in working with aggregate structures \cite{mcdonellTypesafeRuntimeCode}. 
\begin{listing}
    \inputminted{Haskell}{codegen/repr.hs}    
    \caption{The definition of representation types}
    \label{codegen:code:repr}
\end{listing}
\begin{listing}
    \inputminted{Haskell}{codegen/const.hs} 
    \caption{An example usage of reified type in the code generation}
    \label{codegen:code:const}
\end{listing}
Hence, we need to restrict the set of types allowed in SPar. We achieve this using the type class \hask{Repr} and corresponding reified type \hask{ReprType} (shown in \coref{codegen:code:repr}). \hask{Repr} determines the set of type allow in SPar. Reified type \hask{ReprType} will be used to alter the behaviors of code generation based on the type. This can be simply done by pattern matching because reified types are values in Haskell \cite{ReifiedTypeHaskellWiki}. To be more concrete, \coref{codegen:code:const} gives a demo. \hask{constToCExpr} is function that handle code generation from constant value to expression in C programming languages. By pattern matching, we vary the behaviors of code generation so that constants with different types has their own way to be represented in C.

In conclusion, we allow the following type: numerical type like \hask{Float} and \hask{Int}, the unit type \hask{()}, the label type which is used in the code generation of select and branch and the aggregate type: list, product and sum that are built recursively, to be expressed in SPar.

\section{Compilation from SPar to Instr}
% \subsection{Two stages of compilation}
\subsection{Strategies for channel allocation}
\subsection{Type preservation}
\section{Code generation to C: from Instr to C}
\subsection{Representations of Core data type in C}
% \subsubsection{Data type in C}
% \subsection{Optimization for common recursive data types in C}
% \subsection{Memory management}