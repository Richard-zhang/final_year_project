\chapter{Conclusions and future works}
\section{Conclusions}
At the beginning of this project, our goal was to implement a backend for code generation for Alg parallel language using a session-typed intermediate language. The project evolved when we tried to fill the huge gap between Alg computation and SPar communication. It was then we started to design an interface as an abstraction layer on top of SPar for ease of compilation from the high-level language to it. The interface iterated from simple helper functions to arrow interface capturing general computation. Then we realized SArrow should not be limited as a compiler writer tool that hides from users, and it should be exposed to users to express computation. The application as a backend is just a side-product while the project shines as a stand-alone framework for parallel code generation. 

Iterations of the project lead to the current work: A framework generates parallel C code along with local types describing the communication patterns by interpreting data-flow by communication, from Arrow based high-level expressions that can be easily formed, composed and manipulated by users with the help of the hosting function: Haskell.

In conclusion, with the recent release of AMD's latest generation of consumer CPUs featuring a processor with sixteen physical cores, Moore's law will be replaced by the addition of cores. No need to mention the area of high-performance computing where CPU with 64 cores are common. In contrast, most of the programmers have only written sequential code, and most of the algorithms about which students learn are not parallel. We hope this project can contribute its force on parallelism on CPU, encouraging more and more programmer to take advantages of modern computing architectures.
\section{Future works}
There are many interesting future works that we would like to implement. We will select some of them to introduce:
\begin{itemize}
    \item \textbf{Optimization for benchmarks.} Because of the time constraint, there are lots of space to optimize the generated code. We should do more fine-grained profiling on the generated code. It is interesting to use tools like EzTrace to trace and visualize the execution of all the threads. More importantly, reducing the size of the generated code by eliminating common sub-expressions will be useful. At the moment, there are many code duplication for communication among different roles. The only difference is that the role of participating in the communication. The size of generated code can be reduced a lot if we can extract the common part to a function parameterized by the roles participating. 
    \item \textbf{Integrated user experience.} As demonstrated in the evaluation chapter, users need to write the computation using the EDSL in Haskell and then generate c code. From then on, they need to finish the implementation of their atom functions in C. Finally, they can run the generated code with their data in C. The user experience is isolated when you have to write Haskell first and manually completed the generated code and run them in C. Instead, it will be great if we can provide an integrated user experience where the user does everything in Haskell from writing the high-level expression to collect computation results. This is possible thanks to packages like inline-c and foreign language interface in Haskell. User experience will be greatly improve if we can offer an interface in Haskell that looks like \hask{run :: SArrow a b -> (a -> b)}. This function will take a SArrow expression and produces a function that will convert a Haskell value into C data and execute the computation in C and copy back the C output by foreign language interface to Haskell. From the user pointer of way, it can be used the same as a normal Haskell function with type \hask{a -> b}. Forming a closed loop in Haskell would give us the best user interface and automate a large amount of boilerplate work.
    \item \textbf{Fine-grained control for strategies in role allocation.} We talked about how different role allocation strategies give us different parallel computation. It will be great if we parameterize the SArrow with a different strategy and adding ways to specify what strategy will be used at a different stage of the computation. This also opens the possibilities for users to implement their strategies to customize their parallel computation tasks.
    \item \textbf{More customizations. } Similar to customized role allocation strategies, we can even have customized representation of sequential computation since the separation of the communication EDSL and the sequential computation EDSL. This kind of work requires expert usage of Haskell TypeClass and well-designed interfaces and hence is challenging.
\end{itemize}