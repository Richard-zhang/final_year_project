\chapter{SPar: Implementation} \label{chap:impl}
\section{Session types} \label{impl:sec:session}
Haskell does not support session type natively. Other encodings of session types in Haskell is an overkill to this project since Proc does not support all actions that can be typed by session types i.e. channel delegation. We decide to create our representation of session types in Haskell corresponding to set of actions supported by Proc. We will introduce two method for session typing the group of Proc programs followed by duality checking. One is at the value level and another is at the type level.

Essentially, what we do is to infer a collection of local types from a set of SPar expressions. The collection not only can be used to check the duality compatibility of the overall computation but also can work with external tool \cite{langeVerifyingAsynchronousInteractions2019} for additional analysis and verification
\subsection{Representations of session types in Haskell}
Session types belong to the family of behavior types. So the type looks very similar to the value that will be typed. We exploit this similarity to defined our session types in terms of free monad, the same method we have used in defining Proc.
\begin{listing}[ht]
    \inputminted{Haskell}{impl/type.hs}
    \caption{Session types in Haskell}
    \label{impl:code:type}
\end{listing}

The \coref{impl:code:type} shows the definition in Haskell. \hask{SType} is the session type in Haskell. It is parameterized by a type variable \hask{a} so that the value-level session types and the type-level session types can share the same basic definition of session types. \hask{S} is mapped to $! \langle p, S \rangle . T$. \hask{R} is mapped to $?(p, S).T$. \hask{B} is mapped to Branch type and \hask{Se} is mapped to Select type.
\subsection{Value-level duality check}
For the value-level session types, the type variable \hask{a} is instantiated with type \hask{TypeRep}. TypeRep reifies types to some extent by associating type representations to types \cite{DataTypeable}. Due to session types are represented as value expressions in Haskell, session typing a Proc program is the same as writing an interpreter which can be easily done since a Proc program is based on free monad. 

We traverses Proc programs converting each operation to its corresponding type in \hask{STypeF} and convert the value to its typeRep. For output actions, we recursively call the substructure to build the rest of session types. For input actions like \hask{Recv}, we will apply the continuation with the Core value constructed by\hask{Var} and recursively call the function on the result. The trick to apply \hask{Var} to the continuation make it possible to inspect the static structure of every Proc programs because \hask{b -> Proc a} is not inspectable i.e, we cannot pattern match on i, while \hask{Proc a} can be inspected. We will also use this trick in the code generation which will be introduced in later chapter.
\subsection{Type-level duality check} % TODO move to end
The value-level duality check works well in checking duality at runtime but as programmers, we aim to eliminate problems earlier. Hence, we proposed a solution that makes use of Haskell's powerful type systems to check the duality of the system at the compile time. In addition, we will introduce some combinators to helps us build a group of Proc to form parallel computation and this mechanism can act as an extra safety guard to make sure the correctness of these combinators.

The general approach of type-level duality checks can be summarized as the following steps.

\begin{enumerate}
    \item Create a type-level representation of session types.
    \item Modify the algebra of Proc to make it indexed by session types so that we can session type a proc while building it at the same time. Unlike the above method, we  can only session type a Proc by interpretation after it has been constructed.
    \item Gather the indexed session types of each Proc in the system and check the duality pair-wise at the type level.
\end{enumerate}
\begin{listing}[ht]
    \inputminted{Haskell}{impl/typebind.hs}
    \caption{Implementations of type level bind}
    \label{impl:code:typebind}
\end{listing}
The first step is achieved by reusing the definition in \coref{impl:code:type} and use Haskell DataKind extension to promote data constructors of \hask{STypeF} and data constructors of Free monad to type constructors. At this stage, type parameter \hask{a} has been promoted to kind parameter and it will be instantiated with kind \hask{*} representing the kind of all types that have values i.e. Int, List of float. In addition, we should also create type level function that is equivalent to bind in Free monad to help us compose session types. The implementation of type level bind \hask{>*>} can be seen in \coref{impl:code:typebind}. It is similar to bind in Free monad but defined in type family. 
\begin{listing}[ht]
    \inputminted{Haskell}{impl/typeproc.hs}
    \caption{The algebra of Proc indexed by session types and the definition of indexed Proc}
    \label{impl:code:typeproc}
\end{listing}

The second step is challenging since we need to find a way to make the Proc indexed by our free monad. Obviously, the original definition of Free monad and Proc does not provide any extra type parameters to be indexed by session types. Hence we use the indexed free monad. It is indexed by two parameter \hask{i} and \hask{j}. In the context of this project, you can treat \hask{i} as the session type for the current proc  and \hask{j} as the continuation of session types. Accordingly, we will modify the definition of the algebra of Proc as well as Proc (see in \coref{impl:code:typeproc}). The main operations remained unchanged and the main difference is 1) makes the algebra of Proc indexed by its corresponding session type (\hask{i}) and continuation (hask{j}) 2) For the role identifier, we use the type level identifier: the type whose kind is \hask{Nat} instead of values because in the later stage, we have to check duality at the type level. The definition of Proc use Haskell's RankNType and type family \hask{>*>} to extract its corresponding indexed session types \hask{i} from the continuation. By the definition of \hask{>*>}, session type \hask{i} must end with the Pure type constructor which is mapped to end in the session type. The basic helper functions for constructing Proc expression are also indexed by session types. Some example can be in seen in \coref{impl:code:typehelper}. We will omit the implementation for some helpers functions. Observing the function signatures is crucial in understanding how it works.
\begin{listing}[ht]
    \inputminted{Haskell}{impl/typehelper.hs}
    \caption{Implementations of helper functions}
    \label{impl:code:typehelper}
\end{listing}

\begin{listing}[ht]
\begin{minted}{Haskell}
example = do
    x :: Core Int <- recv zero 
    send one x
\end{minted}
\begin{minted}{text}
ghci:> :type example
example
  :: Proc
     ('Free
        ('R 0 Int
            ('Free ('S 1 Int (Pure ())))))
     Int
\end{minted}
    \caption{An example of session type}
    \label{impl:code:typeexample}
\end{listing}

We will conclude the implementation of second step with an example. After all, dependent types in Haskell is not an easy topic. An example is worth thousand words to build understanding (see in \coref{impl:code:typeexample}). It represents a simple proc that receives an int from role zero and send the int to role one. Haskell type system infers its session type is \hask{'Free ('R 0 Int ('Free ('S 1 Int (Pure ()))))} which corresponds to the behavior of the example. Session typing can be done simultaneously and automatically while users are building the Proc processes thanks to Haskell's type inference.

For the third step, we can assume we have already gathered a type level list of session types paired  with its role identifier. The duality check algorithm is still the same. We match different proc pair-wise, and check whether the projection of both session types are complementary. The algorithm is easy to implement as the Haskell function but lifting the computation into type level is tricky. One of the obstacles is that type family does not support higher order type-level functions. We divide the problem into sub-problems. We need to implement 1) a type family that converts a list of Session type to a list of Session type pair 2) a type family that maps a function to list and combine the result 3) a type family that that includes projection, checking whether a pair of session types are complementary. We combine the solutions to these problems and encapsulate them into type class constraint. The constraint is satisfied only if the duality checked is passed at the compile time. 

% In conclusion, 
\section{SPar interpreter} \label{impl:sec:interp}
\subsection{Overview}
SPar interpreter is an interpreter that interprets a group of Proc programs in Haskell. It can be considered as the simplest backend for evaluating SPar expressions. It records traces of the executions and the final output values of each Proc programs in the system. 

It focused on providing a reference implementation explaining the semantics of SPar expressions not on performance. In addition, SPar interpreter served as a very useful tool in the development of this project. We use it as prototype to quickly verify whether the computation produces produces the expected results. This feature is useful especially in the early stage of an implementation or during debugging.
\subsection{Implementation}
The implementation of SPar interpreter is standard. It is similar to the implementation of the scheduler we explained in the free monad section of the background chapter. In essence, it is a round robin scheduler for a group of Proc programs. 
\begin{listing}[ht]
\inputminted{Haskell}{impl/interp.hs}
\caption{Partial implementation of the SPar interpreter} 
\label{impl:code:interp} %TODO cut the implementations
\end{listing}
A partial implementation can be seen in \coref{impl:code:interp}. It takes a list of Proc as a parameter and it maintain a state which is the combination of message queue, the trace and the list of output values. For the base case, the list is empty which means all processes has exited, it returns the trace and the list of output values. For the recursive case, it pattern matches on the first process at the beginning of the list. If the operation is Pure, it will update the list of output values, then abandon this process, and call itself recursively on the tail of the list. If the operation is an output action i.e, \hask{Send}, \hask{Select} or \hask{Broadcast}, it will update the message queue with the corresponding message containing the sender, the receiver and the value, then pop the head proc and append its next step to the end of list of Procs, and call the list recursively. If the operation is an input action i.e, \hask{Recv} or \hask{Branch}, it will first examine whether sender and receiver pairs match the pair from the message at the top of the message queue. If so, it applies the continuation with the value in the message, removes the message from the queue, removes the head of list, moves the result of applying continuation to the end of the list and calls the list recursively. If not, it simply moves the head of list to the end of list without changing the value of head process and call the list recursively. Because we have checked the duality of the processes in the system, this guarantees that for any input action, the required message will eventually appear at the head of the message queue in finite steps.

Also, the interpreter has a helper function that evaluates Core expressions. This helper function is easy to write because operation on product or sum type has its own mapping function in Haskell and for the \hask{Prim} and \hask{Lit}, we can get their underlying Haskell implementation directly. As for \hask{Var}, this constructor is not intended to be used externally so we will not encounter.