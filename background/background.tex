\chapter{Background}
\section{Multiparty session types} 
\subsection{Process calculus}
\subsubsection{Syntax}
\subsubsection{Operation semantics}
\subsection{Type system}
\subsubsection{Global types and session types}
\subsubsection{Projection between types}
\subsection{Example: arithmetic server}
\subsection{Applications in parallel computing}
\section{Parallel algebraic language}
\subsection{Arrows}
\subsection{Applications with session types}
\section{Free monad}
Free monad, a terminology from category theory, is a construction that is left adjoint to a forgetful functor whose domain is the category of Monads and whose co-domain is the category of Endofunctors \cite{contributorsCatsFreeMonads}. Intuitively, it says a monad can be constructed from any functor. Functional programming language exploit this concept because of its applications in domain specific language, especially in interpreting (section \ref{b:fm:a}).
\subsection{Definition}
In practice, free monad in Haskell can be defined as a an algebraic data type.
% \begin{lstlisting}[frame=single, language=Haskell]
% data Free f a =
%     Pure a
%   | Free f (Free f a)
% \end{lstlisting}
\begin{minted}{haskell}
data Free f a =
    Pure a
  | Free f (Free f a)
\end{minted}
\subsection{Example}
    
\subsection{Applications} \label{b:fm:a}